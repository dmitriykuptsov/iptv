\documentclass[conference,10pt,letter]{IEEEtran}

\usepackage{url}
\usepackage{amssymb,amsthm}
\usepackage{graphicx,color}

%\usepackage{balance}

\usepackage{cite}
\usepackage{amsmath}
\usepackage{amssymb}

\usepackage{color, colortbl}
\usepackage{times}
\usepackage{caption}
\usepackage{rotating}
\usepackage{subcaption}

\newtheorem{theorem}{Theorem}
\newtheorem{example}{Example}
\newtheorem{definition}{Definition}
\newtheorem{lemma}{Lemma}

\newcommand{\XXXnote}[1]{{\bf\color{red} XXX: #1}}
\newcommand{\YYYnote}[1]{{\bf\color{red} YYY: #1}}
\newcommand*{\etal}{{\it et al.}}

\newcommand{\eat}[1]{}
\newcommand{\bi}{\begin{itemize}}
\newcommand{\ei}{\end{itemize}}
\newcommand{\im}{\item}
\newcommand{\eg}{{\it e.g.}\xspace}
\newcommand{\ie}{{\it i.e.}\xspace}
\newcommand{\etc}{{\it etc.}\xspace}

\def\P{\mathop{\mathsf{P}}}
\def\E{\mathop{\mathsf{E}}}

\begin{document}
\sloppy
\title{Building IPTV streaming service using commodity hardware}
\maketitle
\begin{abstract}

  The purpose of this document is to get acquainted with the theory and practice behind 
  IPTV streaming services. We hope that the knowledge obtained during the study will be
  essential for those who will be maintaining and troubleshooting small scale IPTV 
  infrastructure.

  We start this short document with the discussion of the theory behind digital video broadcasting
  services, including various modulation schemes, audio and video codecs, error correction algorithms,
  and other related issues. We then move on to the discussion related to design and implementation of a simple IPTV streaming service using commodity hardware. Finally, to understand the scalability of the proposed architecture, we will perform a set of simple experiments (the experiments will mostly focus on microbanchmarking of various components of the system).

\end{abstract}
\input intro.tex
\input background.tex
\input hardware.tex
\input experimental.tex
\input conclusions.tex
%\balance
\bibliographystyle{abbrv}
\bibliography{mybib}

\end{document}

